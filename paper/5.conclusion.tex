% 5.conclusion.tex - Conclusion
\section{Conclusion}
\label{sec:conclusion}

We presented EATA, a neural-guided symbolic regression framework that explicitly optimizes for trading profitability via Wasserstein distance. Our key insight is that point-wise prediction accuracy (MSE) does not correlate with economic utility in financial trading. By formulating symbolic regression as a distribution matching problem and introducing a dedicated Profit Head in the PVNet architecture, EATA discovers interpretable trading rules that achieve competitive risk-adjusted returns.

We formalize trading as a \textbf{distribution matching problem} using Wasserstein distance, providing theoretical justification for why this metric captures tail risk and directional asymmetry better than MSE or KL divergence. Building on this objective, we develop a \textbf{three-headed PVNet} (Policy-Value-Profit) and show that explicit profit-aware learning yields substantially higher risk-adjusted returns than an accuracy-only NoRL variant, improving Sharpe ratio from 0.66 to 0.83 under the same experimental protocol. Finally, we conduct \textbf{empirical validation} on representative S\&P 500 constituents over 4 years, with in-depth ablations confirming the necessity of the Profit Head and domain-specific grammar, and we provide an executable pipeline to support the planned full-universe evaluation.

Several limitations remain. \textit{Computational cost}: MCTS incurs substantial runtime per stock relative to gradient-based baselines, which constrains applicability to high-frequency or large-universe settings. \textit{Grammar design}: the fixed-window operator set still relies on domain expertise for window selection, and automated discovery or adaptation of grammars is left for future work. \textit{Single-asset focus}: the current implementation models each stock independently and does not account for portfolio-level interactions or cross-asset structure. \textit{Extreme regimes}: although sliding windows offer some robustness to moderate non-stationarity, abrupt exogenous shocks can still invalidate historically learned symbolic patterns.

Future work will focus on four complementary directions. First, we aim to improve \textit{scalability} by exploring parallel MCTS implementations and GPU-accelerated expression evaluation, making the framework more suitable for large universes and finer time resolutions. Second, we plan a \textit{portfolio-level extension} in which the search space explicitly includes cross-asset operators, enabling the discovery of expressions that model joint dynamics and support portfolio optimization rather than single-asset trading in isolation. Third, we will investigate \textit{online and continual learning} schemes that update symbolic expressions incrementally as new data arrive, reducing the need for full retraining and better accommodating regime shifts. Finally, we are interested in integrating \textit{causal discovery} tools to distinguish robust, structurally meaningful relationships from spurious correlations, thereby further enhancing the reliability and interpretability of the learned trading rules.

Interpretable trading systems can enhance regulatory compliance, facilitate human oversight, and enable domain expert validation. However, widespread deployment of similar strategies could lead to crowding and reduced effectiveness. Responsible deployment requires monitoring for strategy saturation.

